\documentclass[dvipdfmx,uplatex]{jsarticle}

\usepackage{color,amsmath,amssymb,enumerate,graphicx}
\usepackage[top=30truemm,bottom=30truemm,left=25truemm,right=25truemm]{geometry}
\usepackage{mysty}
\usepackage{mythmj}
\usepackage{tcolorbox}

% \tcbuselibrary{skins}%shadow
\tcbuselibrary{raster,skins}
\newtcolorbox{defbox}{enhanced, colback=white, colframe=black, sharp corners, boxrule=.75pt, drop shadow=black!30!white}
\newtcolorbox{theorembox}{enhanced, colback=white, colframe=black, boxrule=.75pt, arc=2mm, drop shadow=black!30!white}

\begin{document}

\title{Mathematics}
\author{\texttt{miruca}}
\date{\today}
\maketitle

\section{概要}
これは \LaTeX が Visual Studio Code でビルドできるかどうかのテスト文章です.

\section{統計}
データ$x_1,x_2,\dots,x_n$が時間的に観測されたものであるとき,これらを一般に\textbf{時系列}(time series)という.

\section{基本統計量}
データを$x_1,x_2,\dots,x_n$とする.平均とは,$\overline{x}:=(x_1 + x_2 + \dots + x_n)/n$で定義される.
分散とは,$\sigma^2:=\mathrm{E}\left( \left( X - \mathrm{E}(X) \right)^2 \right)$で定義される.
標準偏差とは,$s:=\sigma$で表される.すなわち,分散の\texttt{sqrt}をとったものである.

\section{数式}
\begin{align}
    \sin(\alpha \pm \beta) &= \sin \alpha \cos \beta \pm \cos \alpha \sin \beta \\
    \cos(\alpha \pm \beta) &= \cos \alpha \cos \beta \mp \sin \alpha \sin \beta \\
    \tan(\alpha \pm \beta) &= \frac{\tan \alpha \pm \tan \beta}{1 \mp \tan \alpha \tan \beta}
\end{align}

\begin{defbox}
    \begin{definition}
      hoge
    \end{definition}
\end{defbox}

\begin{proof}
    prooooof.  
\end{proof}
%
\vspace{\baselineskip}
\begin{theorembox}
    \begin{theorem}
      hoge
    \end{theorem}
\end{theorembox}

\end{document}